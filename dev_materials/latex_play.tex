\documentclass[border=1cm]{standalone}
\usepackage{tikz}
\usepackage{xcolor}
\begin{document}
\usetikzlibrary{shapes,arrows}
%\begin{tikzpicture}
%  \foreach \x /\alph/\name in {0/a/History, 51/b/Why?, 103/c/Bayes\\Theorem, 154/d/Philosophy, 206/e/Priors and\\Probabilities, 257/f/Properties, 309/g/Sample-Base\\Approximation}{
%    \node[circle, fill=green,minimum width=15mm, draw, font=\tiny, align=center,shading=axis,top color=orange,bottom color =orange!50!black] (\alph) at (\x:3cm) {\name}; }
%
%  \foreach \alpha in {a,b,c,d,e,f,g}%
%           {%
%             \foreach \alphb in {a,b,c,d,e,f,g}
%                      {
%                        \draw (\alpha) -- (\alphb);%
%                      }
%           }
%\end{tikzpicture}
foobar \\
\begin{tikzpicture}
  \node at (0,0) [rectangle, fill=blue, draw, text width=4.5em, align=center,shading=axis](likelihood){observed data};
  \node[circle, fill=blue, draw, text width=4.5em, align=center,shading=axis, left of = likelihood](prior){Before data};
  \node[circle, fill=blue, draw, text width=4.5em, align=center,shading=axis, right of = likelihood](posterior){After data};
  \path [draw, -latex'] (prior) -- (likelihood); %
  \path [draw, -latex'] (likelihood) -- (posterior);
  \node[rectangle, draw, text width=4.5em, align=center,below of=likelihood,opacity=0.01,text opacity=1](likelabel){Likelihood};
  \node[rectangle, draw, text width=4.5em, align=center,below of=prior,opacity=0.01,text opacity=1](priorlabel){Prior};
  \node[rectangle, draw, text width=4.5em, align=center,below of=posterior,opacity=0.01,text opacity=1](postlabel){Posterior};    
\end{tikzpicture}

\end{document}
        

            
